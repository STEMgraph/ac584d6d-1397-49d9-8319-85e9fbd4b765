learningobjective{}
\begin{challenge}
    \chatitle{Basic operations from the terminal} 
    \begin{chadescription}
        The usage of interactive command-line interperters started in the 1960s.
        By hooking up a teletype 33 terminal to a computer, a person could now interact with the operating system by typing characters and sending them to an input buffer.
        A series of bits was sent by the teletype and received by a \href{https://github.com/STEMgraph/}{deserializer circuit} in the computer, to store it in an input buffer.
        The computer then read from this input-buffer did it's job and eventually stored a gererated message in the output buffer.
        Just like the magical computer from \href{https://github.com/STEMgraph/}{Magic-Computer Challenge}.
        The output buffer was then transmitted to the teletype by a \href{https://github.com/STEMgraph/}{serializer circuit}.
        This goes back and forth. 
        Today we don't use teletype-terminals anymore, but so called \textit{terminal-emulators}.
        Yet the terminal-emulator has the same function as the teletype before. 
        In this challenge you will learn how to use the computer to do computations for you by handling input and output through the terminal-emulator.
    \end{chadescription}

    \begin{task}
        Open up your terminal emulator and proceed with the following steps.
        The most basic thing a computer should be able to do, is to compute something for you.
        So lets go ahead and evaluate some simple expressions.
        These expressions, are being sent by the terminal emulator to the \textit{shell}-program, then passed on to the hardware by the operating system, passed back to the \textit{shell}-program, and finally, to the terminal emulator.
        \begin{questions}
            \item Type \texttt{expr 1 + 2} and press \texttt{Enter}, Make sure, that you correctely typed the expression including the spaces.
            \item Type \texttt{expr 3 - 2} and press \texttt{Enter}.
            \item Type \texttt{expr 3 \* 2} and press \texttt{Enter}.
            \item Type \texttt{expr 1 / 2} and press \texttt{Enter}.
            \item Type \texttt{expr 2 / 1} and press \texttt{Enter}.
            \item Type \texttt{expr 2 \* 2} and press \texttt{Enter}.
            \item Type \texttt{expr 3 \* 2} and press \texttt{Enter}.
        \end{questions}
    \end{task}

    \begin{task}
        Next to the arithmetical operators, the \textit{shell}-program also includes logical operators.
        Let's take a look at them.
        \begin{questions}
            \item Type \texttt{expr 3 = 3} and press \texttt{Enter}, Make sure, that you correctely typed the expression including the spaces.
            \item Type \texttt{expr 3 != 3} and press \texttt{Enter}.
            \item Type \texttt{expr 2 \> 3} and press \texttt{Enter}.
            \item Type \texttt{expr 3 \< 6} and press \texttt{Enter}.
            \item Type \texttt{expr 3 \>= 3} and press \texttt{Enter}.
            \item Type \texttt{expr 3 \<= 3} and press \texttt{Enter}.
        \end{questions}
    \end{task}

    \begin{task}
        In order to calculate more complex expressions, we may also use parentheses.
        Let's take a look at some of them.
        \begin{questions}
            \item Type \texttt{expr \( 3 + 2 \) \* 2} and press \texttt{Enter}, Make sure, that you correctely typed the expression including the spaces.
            \item Let's figure out, whether an expression is correct or not: Type \texttt{expr \( 3 + 2 \) \* 2 = 10} and press \texttt{Enter}.
        \end{questions}
    \end{task}

    \begin{task}
        Just for the record: there are also expressional evaluations for strings.
        Here are some examples.
        \begin{questions}
            \item Type \texttt{expr "hello" = "hello"} and press \texttt{Enter}.
            \item Type \texttt{expr "hello" != "hello"} and press \texttt{Enter}.
            \item Type \texttt{expr length "hello"} and press \texttt{Enter}. 
            \item Type \texttt{expr length "hello" = 5} and press \texttt{Enter}.
            \item Let's find the first occurrence of a character in a string by using the expression \texttt{index <string> <character>}: Type \texttt{expr index "hello" "e"} and press \texttt{Enter}.
            \item We can also find the last occurrence of a character in a string by using the expression \texttt{rindex <string> <character>}: Type \texttt{expr rindex "terminal emulator" "e"} and press \texttt{Enter}.
            \item To cut a string, we use the expression \texttt{substr <string> <start> <length>}: Type \texttt{expr substr "terminal-emulator" 3 5} and press \texttt{Enter}.
            \item Now write an expression, that checks whether a substring starting at a certain position is equal to a certain string. Type \texttt{expr substr "terminal-emulator" 3 5 = "emulat"} and press \texttt{Enter}.
        \end{questions}
    \end{task}

    \begin{advise}
        Even though we'll learn way more powerful tools to handle data, it is always important to make sure, that you know the basics concepts. 
        Go over all of the tasks a second time and vary the arguments. 
        Make sure, you can predict the output of the commands.
        Also to provoke errors with these commands and try to understand the output.
    \end{advise}
\end{challenge}
    
